\documentclass[modern]{aastex62}

% lsstdoc documentation: https://lsst-texmf.lsst.io/lsstdoc.html
\input{meta}

% Package imports go here.

% Local commands go here.



\newcommand{\docRef}{PSTN-035}
\newcommand{\docUpstreamLocation}{\url{https://github.com/lsst-pst/pstn-035}}


\begin{document}
\input{authors}
\date{\today}
\title{Integration, Test and Commissioning Results from LSST Commissioning Camera}
\hypersetup{pdftitle={\@title}, pdfauthor={\@author}, pdfkeywords={\@keywords}}


\begin{abstract}

The Vera C. Rubin Observatory construction project involves assembly, integration, and verification testing phases.  Prior to the installation of the 3.2 Gpix LSSTCam, the Rubin Observatory Commissioning Camera (ComCam) will test the functionality and performance of the telescope, facility, and various software and DM/IT infrastructural components, although at a smaller scale than the final LSSTCam configuration which immediately follows this inital 3 month on-sky campaign.  During this time, the ComCam observations will verify most of the camera interfaces as well as demonstrate the performance of the Simonyi Telescope Active Optics System.

\end{abstract}


The LSST construction project involves assembly, integration, and verification testing phases.  Prior to the installation of the 3.2 Gpix LSSTCam, the Commissioning Camera (ComCam) will test the functionality and performance of the telescope, facility, and various software and DM/IT infrastructural components, although at a smaller scale than the final LSSTCam configuration following this 3 month on-sky campaign.

\section{Introduction}

Provide background of LSST, science cases, and provide context for Comcam (system verification, performance characterization, and ICD verification.

\section{mechanics}

Overall layout with solid model (or actual photographs) figures showing the components.

\subsection{cryostat}

The Emony BNL cryostat

\subsection{utility trunk}

The quadbox design

\subsection{sled}

mounting platform for cryostat

\subsection{mass simulator}

physical surrogate for LSSTCam

\section{camera, electronics}

Details on the camera components

\subsection{CCDs}

The sensors

\subsection{Readout Electronics}

The REBs

\subsection{DAQ/CCS}

The data acquisition and camera control systems.

\subsection{cryotels/thermal}

Cryotels, thermal design description

\section{optics}

description of the T\&S components

\subsection{field flattener}

3 lens design

\subsection{filters}

LSSTCam witness samples

\subsection{shutter}

photometric shutter

\subsection{baffle}

light rejecting design

\section{refrigeration pathfinder}

General overview of the refrigeration system

%\subsection{initial refrigeration testing}

%planned activities for preparing and characterizing the refrigeration system

\section{internal performance}

overall camera performance (EO testing and internal mechanical tolerance)

\subsection{camera}

shutter, filters, readout time

\subsection{CCDs}

badpix, gain, read noise, CTE

\subsection{optical alignment}

tolerance budget, best measurements, AR coating decision

\subsection{mechanical alignment}

rotator-optics alignment, flexure

\section{on sky performance}

Description of activities while ComCam is installed on the telescope.

\subsection{laser tracker alignment}

positioning precision of the optics, measured flexture.

\subsection{in-dome calibration}

flat field, CBP data.

\subsection{photometric quality}

star photometry in selected fields

\subsection{image quality}

psf image quality in selected fields

\section{conclusions}

take away for an astronomical camera, and evaluating a 8.4m survey telescope

%\subsection{commissioning schedule}

%breakdown of time allocation for each commissioning activity (system performance, AOS tuning, DM testing, etc)




\section{Introduction}

Provide background of LSST, science cases, and provide context for Comcam (system verification, performance characterization, and ICD verification.


\section{mechanics}

Overall layout with solid model (or actual photographs) figures showing the components.

\subsection{cryostat}

The Emory BNL cryostat

\subsection{utility trunk}

The quadbox design

\subsection{sled}

mounting platform for cryostat

\subsection{mass simulator}

physical surrogate for LSSTCam


\section{camera, electronics}

Details on the camera components

\subsection{CCDs}

The sensors

\subsection{Readout Electronics}

The REBs

\subsection{DAQ/CCS}

The data acquisition and camera control systems.

\subsection{cryotels/thermal}

Cryotels, thermal design description

\section{optics}

description of the T&S components

\subsection{field flattener}

3 lens design

\subsection{filters}

LSSTCam witness samples

\subsection{shutter}

photometric shutter

\subsection{baffle}

light rejecting design

\section{refrigeration pathfinder}

General overview of the refrigeration system

\subsection{initial refrigeration testing}

planned activities for preparing and characterizing the refrigeration system

\section{performance}

overall camera performance

\subsection{camera}

shutter, filters, readout time

\subsection{CCDs}

badpix, gain, read noise, CTE

\subsection{optical alignment}

tolerance budget, best measurements, AR coating decision

\subsection{mechanical alignment}

rotator-optics alignment, flexure

\section{conclusions}

take away for an astronomical camera, and evaluating a 8.4m survey telescope

\subsection{commissioning schedule}

breakdown of time allocation for each commissioning activity (system performance, AOS tuning, DM testing, etc)


\appendix
% Remove this when you strart your paper
\input{appendix}
% Include all the relevant bib files.
% https://lsst-texmf.lsst.io/lsstdoc.html#bibliographies
\section{References} \label{sec:bib}
\bibliographystyle{yahapj}
\bibliography{local,lsst,lsst-dm,refs_ads,refs,books}

% Make sure lsst-texmf/bin/generateAcronyms.py is in your path
\section{Acronyms} \label{sec:acronyms}
\input{acronyms.tex}

\end{document}
